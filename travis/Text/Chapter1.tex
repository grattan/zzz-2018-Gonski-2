\chapter{The context: new national school reforms are likely in 2018}\label{chap:National-reform-context}

The Gonski 2.0 Review comes at a critical time. Australia’s educational performance is declining internationally, we face new challenges in preparing students for future work, and equity gaps are too wide. 

The time is ripe for a discussion on the Commonwealth Government’s role as it negotiates a new agreement on school funding with the states and territories later this year. 

This report argues that the Commonwealth should not have a much bigger role in schooling than it does today. Federal Government over-reach could do considerable damage. 

\section{The Gonski 2.0 Review will help inform the Commonwealth's next steps}\label{sec:Commonwealth-next-steps}

The Turnbull Government wants to ensure that its promised extra \$23 billion of schools funding is spent wisely. The extra federal money was announced in 2017 as part of new funding arrangements (known as `Gonski 2.0') that seek to better align funding to student need. The Commonwealth and state and territory governments are currently negotiating the terms of the new funding agreement that spans over ten years from 2018-2027.

The Gonski 2.0 Review has been commissioned in this context. The review team has been asked to examine the evidence on what works in schools and school systems, as well institutional, governance, transparency and accountability measures required to ensure what works is implemented. The review team has received numerous public submissions.\footnote{All submissions, including Grattan Institute's, are due to be made public once the Gonski 2.0 Review team has delivered its final report.}

\section{New federal conditions on funding could be on the table}\label{sec:Commonwealth-influence}

Under constitutional arrangements, state and territory governments are responsible for ensuring the delivery and regulation of schooling.

However the federal government can exert greater control over schooling policy through Section~96 of the Constitution which allows for conditions on Commonwealth funding to state and territory governments. 

The current federal government has sent some signals that it could seek to use the Gonski 2.0 Review findings to impose new conditions and increase its influence over school education policy.

The federal Department of Education and Training has stated that the Review will `\textit{make sure that reform actions are based on a solid understanding of what works}' and that `\textit{implementation of reforms will be a condition of funding for states}'.\footcite{DETQualitySchoolsFrequentlyAskedQuestions}

And the Commonwealth’s 2016 strategic document Quality Schools, Quality Outcomes includes about 15 new input and output reforms that state and territory governments could be required to implement.\footcite{2016AustralianGovernmentQualitySchoolsQualityOutcomes} 
For example, it suggests focusing on reforms `requiring teachers to use explicit literacy and numeracy instruction in schools' (the list of reforms is included in \Chapref{chap:New_initiatives_quality_schools}).\footnote{The status of these reforms is still unclear, because the Education Council typically needs to agree for them to have standing.}

\section{New requirements are not desirable}\label{sec:Commonwealth-actions-can-have-unintended-consequences}

If the Commonwealth does choose to impose new input and output conditions, this would be a significant departure from recent approaches to Commonwealth-state relations and would run counter to the learnings from the 2014 White Paper series on federal reform.\footnote{The White Paper process documented learnings from past Commonwealth-state reform efforts and explored ways to improve federal financial relations in Australia.} 

Schools and states should be held accountable for students' educational progress and ensuring that money is spent wisely.\footnote{This refers not only to government education departments but also Catholic and independent schools.}
But the Commonwealth should tread warily when seeking to increase accountability to themselves rather than the public. If federal policy makers pull the wrong levers, the consequences can be very damaging. 

The Commonwealth should keep its desire to expand control in check. The promised extra Commonwealth funds are only 3 per cent of all government spending on schools from 2018 to 2027. It is important to ensure that \textit{all} government money invested in schools is well spent – and most of that money is provided by state and territory governments.

The overall reform agenda, for which states and territories are primarily responsible, should focus on the changes that will really shift the dial, rather than the sub-set of issues that the Commonwealth can achieve.

\section{The structure of this report}\label{sec:The-big-national-conversation}

The next two chapters explain the big reforms in school education in Australia, many of which are within state and territory responsibilities. \Chapref{chap:Design_an_adaptive_system_of_continuous_improvement} describes the adaptive system design settings needed to achieve continuous improvement in schools, and highlights the need for more systemic support to help frontline professionals embed evidence in daily practice. \Chapref{chap:Specific-reforms} identifies specific system reforms that could make a big difference, drawing on previous Grattan Institute work. 

The remaining chapters discuss what role the Commonwealth should play in this broader reform agenda. \Chapref{chap:Commonwealth-conditions} argues the Commonwealth should refrain from prescriptive new conditions. \Chapref{chap:few-reforms} explains how the Commonwealth should prioritise only a small number of national reforms, in close collaboration with states and territories. \Chapref{chap:what_com_should_do} recommends four specific national reforms the Commonwealth should collaborate on to benefit students across the nation.

Lastly, Early childhood education is outside of the scope of the Gonski 2.0 review and not discussed in this report, but should be a high priority for national reform given its big impact on student outcomes, especially for disadvantaged children. Many children are behind or at risk when they start school, and many never catch-up. 


