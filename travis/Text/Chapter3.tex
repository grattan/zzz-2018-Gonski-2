\chapter{Big reforms will deliver real improvements}\label{chap:Specific-reforms}

This chapter outlines a number of big reforms at the system level that will help embed the use of evidence in schools. They include focusing more on student progress (growth) rather than achievement at a point in time, and improving teaching effectiveness and school leadership. Policy makers also need to gather better data on what is actually happening inside schools.

Critically, most of the big reforms are in areas of state and territory government responsibilities. This is not surprising given they are the system managers of schools.

\section{Focus more on student progress (growth)}\label{sec:Focus-more-on-student-progress-growth}

School education policy should explicitly aim to improve the progress (growth) of all students, not just their achievement at a point in time. This requires:

\begin{itemize}
    \item Putting more `small data' on student progress in the hands of teachers so they can improve teaching in the classroom; and
    \item Putting better `big data' on student progress in the hands of policy makers so they can monitor the system more effectively.
    
\end{itemize}

Small data should be the first priority, because this is known to have one of the biggest impacts on the effectiveness of teaching. Teachers should better use classroom data to track the progress of each of their students, and adapt their teaching to suit what each student is ready to learn next. 

Unfortunately, this is not the norm in Australian schools, as we discussed in our 2015 report, Targeted Teaching.\footcite{Goss2015TargetedTeachingHow}
Student achievement varies by up to seven years in a typical Year 9 classroom in Australia.\footcite{Goss2016Wideninggapswhat}
To better use data in practice, teachers need more tools, training, trust, time and team work (as discussed in \Vref{subsec:strengthen_the_use_assessment}).

Australia must improve how it measures student progress both on core academic skills and `new' capabilities such as critical thinking and non-cognitive skills (the latter issue is discussed in \Chapref{chap:what_com_should_do}). 

\section{Improve teaching effectiveness}\label{sec:Improve-teaching-effectiveness}

Effective teaching has the largest impact on student learning outside of the home environment.\footnote{Discussed in \textcite{Jensen2010Investinginourteachers}.} But too often we talk about teacher quality as though the individual teacher is the point at issue. No teacher is an island; teachers need more support from the system. Six specific proposals are suggested in the following sections, drawing on past Grattan reports. 

\subsection{Prioritise teacher time and standardise some elements of practice}\label{subsec:prioritise} 

Teacher time is an expensive and precious resource. But simply giving teachers more time will not necessarily lead to better teaching and learning. Teacher time must be redirected from low-impact to high-impact activities. This means relieving teachers of administrative activities. But it also means more standardisation of daily teaching practices.\footnote{The first issue is discussed in \textcite{Jensen2012Makingtimeforgreatteaching} and the second issue in \textcite{Goss2015TargetedTeachingHow}.} 

In particular, more use of high-quality, tried and tested support materials can enhance student learning and reduce `reinvention of the wheel'. Standardisation could include more common lesson plans and formative assessments, more guidance on which textbooks to use and how to use them, and careful use of educational technology. 

Policy makers and school leaders must lift their game: they make many of the critical decisions on teachers' use of time.  

\subsection{Strengthen the use of assessment data}\label{subsec:strengthen_the_use_assessment}
Three changes should be made to help teachers use data more effectively to improve their teaching.\footnote{Discussed in \textcite{Goss2015TargetedTeachingHow}.}

First, teachers should get better guidance on how to interpret data on student progress and then adapt their teaching. This is not just about more `data managers' in schools, but more specialised pedagogical guidance to help translate the data into instructional steps. It’s the dialogue that matters, not the data.

Second, teachers should get more and better classroom assessment tools and resources. Such tools should be aligned to the curriculum, easy to use, and provide guidance on how to use data to adjust teaching. Australia needs better tools to measure foundational skills in all subject domains (not just literacy and numeracy) as well as `new' capabilities in critical thinking and non-cognitive outcomes.

Third, government should help teachers to evaluate the tools available, for example by introducing a `star rating' system. Too often, schools and teachers choose their tools based on trial and error, anecdote or a Google search.

\subsection{Make collaborative learning more productive}\label{subsec:collaborative-learning}

Collective teacher efficacy has one of the largest effects on student learning.\footcite{Hattie2015Whatworksbestineducation}
But simply working in a group is not enough, as seen in the US where there have been huge investments with little returns.\footcite{TNTP2015TheMirage}
Australia needs much better ways for teachers to collaborate, moving beyond the simple exchanging of lesson plans to deeper discussions on instruction, interpreting data, and integrating evidence into new ways of working. High-performing education systems such as Shanghai, Singapore and Hong Kong show the way, with their focus on `professional learning communities', including valuable input from expert teachers who guide group discussions.\footnote{Discussed in \textcite{Jensen2012CatchingUpLearning}.}   

\subsection{Improve feedback and appraisal}\label{subsec:feedback}

Feedback is one of the most powerful ways to improve teaching practice.\footnote{Discussed in \textcite{Jensen2011BetterTeacherAppraisal}.}   And it doesn't cost much. Teachers need feedback about their strengths and weaknesses if they are to improve their teaching. But many in Australia don’t get that information during professional learning, appraisal, or their performance management.

\subsection{Strengthen student engagement}\label{subsec:engagement}

As many as 40 per cent of students in Australia are unproductive in a given year, and these students learn less over time.\footnote{Discussed in \textcite{Gossetal2017Engagingstudents}.}
Teachers find this very stressful and are calling out for more support. We must provide better initial training and in-school support in managing classes, as well as better research on the root causes of the problem.
\subsection{Redesign the workforce so that top teachers spread best practice}\label{subsec:specialist-teachers}

Our best teachers can help lift the effectiveness of the whole workforce. Yet they often remain isolated, with heavy teaching loads in their own classrooms.\footnote{Discussed in \textcite{Pete-2016-theOz-School-funding-circuit-breaker-with-appeal-to-all}.}
In high-performing systems, such as Shanghai and Singapore, an elite cohort of specialist teachers sets the direction for effective practice and spreads the message via cross-school networks.\footnote{Discussed in \textcite{Jensen2012CatchingUpLearning}.}  

Australia’s top teachers have some of these functions on paper, but they rarely get to enact them in their schools, or across schools.\footnote{Some Australian states have more stringent policies on how senior teachers are used in schools; for example, Queensland's new `master teachers' are expected to develop teachers across schools.} While the Australian Professional Standards for Teachers have been an important step in introducing the roles of Highly Accomplished and Lead Teachers (HALT) who develop other teachers in schools, these roles are not the norm in most schools today.

State and territories should introduce new `master teacher' and `expert teacher' roles for teachers who are specialists in their subject areas to address this issue. Such positions would not only build workforce capacity but elevate the importance of subject-specific teaching expertise.\footnote{These positions have two important differences to the roles of Highly Accomplished and Lead Teachers (HALT) in the Australian professional standards: 1) they are both subject-specific, and 2) the master teacher role works across schools. Ideally these features should be embedded in the HALT role descriptions in the national standards.}

\section{Revamp school leadership pathways}\label{sec:revamp-leadership}

School leaders are critical to school improvement, yet Australia doesn't select or properly train people well for these roles. School principal shortages will become much worse unless the career path is made clearer and more attractive. Singapore is a shining example: it identifies outstanding leaders early,  provides them with intensive training (a six-month, full-time program), and follows up with strong peer-network support.\footnote{Discussed in \textcite{Jensen2012CatchingUpLearning}.}

\section{Strengthen the evidence base and data flows}\label{sec:evidence-base}

Australia needs to improve how it produces and disseminates evidence on what works in classrooms.\footnote{Discussed in \textcite{Goss2016SubmissiontotheProductivityCommissionInquiryintotheNationalEducationEvidenceBase}.}
In particular, we need to:
\begin{itemize}
    \item Lift the standards for scientific evidence, and produce more randomised controlled trials and quasi-experimental studies. Major government policies should be better evaluated, and more funding provided for longitudinal studies to identify trends over time. Establishing nationally agreed scientific evidence standards would be a good first step. 
    \item Conduct better research on the conditions that encourage teachers to use the evidence on what works.
    \item Better synthesise, translate and share research findings so they are readily accessible to educators and policy makers across the country.
    \item Build the research capacity of schools and policy makers through specialised training and support
    
\end{itemize}

In addition, the network of state-based education research institutions should be strengthened. Research could be shared more widely across states, along with better national coordination of major research efforts to avoid duplication and gaps. 

The analytic capabilities of state based research bodies could also be improved. For example the NSW Centre for Education Statistics and Evaluation (CESE) was established in 2012 to improve the effectiveness, efficiency and accountability of education in NSW\@. Its work informs education funding in NSW, by determining what works and where investment will have the most impact. 

\subsection{Better understand what is happening in schools}\label{subsec:practices}

Education policy makers cannot ensure money is spent well if they do not know what is actually happening in schools. In Australia, too little is known about which pedagogical methods are being used, or the nature of collaboration in schools. As a consequence it is difficult to identify problems, formulate solutions and evaluate results.\footnote{The Victorian Auditor-General has raised this issue, see \textcite{2010Auditor-General}.}   

\section{Emerging reforms to explore}\label{sec:emerging}

The following reforms could hold promise and should be explored further:

\begin{itemize}
    \item Redesigning Initial Teacher Education, so fewer people are trained more intensively (as done in Singapore).\footnote{Discussed in \textcite{Jensen2012CatchingUpLearning}.} This could produce better outcomes for no extra cost. 
    \item Increasing the use of high-quality textbooks and programs of curriculum content, so individual teachers do not have to `reinvent the wheel'.\footcite{Koedel2017Bigbangforjustafewbucks} 
    \item Expanding the use of student feedback, which has been shown to be a reliable indicator of teaching quality.\footcite{MET2012AskingStudentsAboutTeaching}
    \item Training teachers in how to use technology to enhance their teaching, especially in subjects such as maths where there could be large benefits.
    \item Tackling teacher shortages in maths, science and IT, through salary increases and by training existing teachers to give them specialist capabilities.\footcite{ProductivityCommission2017Shiftingthedial}
    \item Boosting incentives to attract high-performing teachers to disadvantaged schools.\footnote{Discussed in \textcite{Riceetal2017Howtogetqualityteachersindisadvantagedschools}.}
    
\end{itemize}


\section{What governments should not do}\label{sec:governments-not-do}

We recommend against an even greater focus on targets, or teaching standards and regulations. 
Targets are useful in agreeing on priorities, but can divert resources from important but less visible activities. Similarly, teaching standards and regulations help to guarantee minimum quality but are unlikely to significantly enhance workforce development.

It is also dangerous to rely too much on school autonomy, transparency, accountability and choice as key levers for improvement.\footnote{Discussed in \textcite{Jensen2013Themythsofmarkets}.}
In particular, the evidence shows that increasing school autonomy will not get the desired results in the absence of the right system support for schools.\footcite{Suggett2015SchoolautonomyNecessarybutnotsufficient}

