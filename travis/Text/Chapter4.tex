\chapter{New Commonwealth conditions are not the way to go}\label{chap:Commonwealth-conditions}

This chapter argues that the Commonwealth should have only a modest role in the big reform agenda to improve school education. It should refrain from expanding its control over school spending. Prescriptive Commonwealth funding conditions have been tried before for little benefit. They can create confusion and tend to result in tick-a-box responses. 

\section{We must learn from history}\label{sec:learn-from-history}

The 2008 Council of Australian Governments (COAG) reform agenda was an attempt to leave behind the old bureaucratic processes. It produced the National Education Agreement, which introduced national performance benchmarking, professional teaching and leadership standards, and a national curriculum. A number of these initiatives were successful, and many were rigorously evaluated, adding to the knowledge base on what works. But there were many costs related to input and output control elements that outweighed the benefits.\footnote{The COAG reform agenda was a shift to outcomes-based federal conditions, however as the reform agenda rolled out there was more of a focus on inputs and processes than intended.}

The 2014 Federation White Paper process documented some of these failings.%
\footcite{2014AustralianGovernmentReformoftheFederationWhitePaper}
It called for a better allocation of roles and responsibilities to make it easier for governments to identify what the problems are in education and who is responsible for fixing them -- and for the public to identify who is accountable.\footcite{2014AustralianGovernmentReformoftheFederationWhitePaper}
It said confusion about accountability can arise where state and territory governments focus on reporting to the Commonwealth rather than to their constituents.

The White Paper series pointed to increased red tape and wasteful duplication of effort.\footcite{2014AustralianGovernmentReformoftheFederationWhitePaper} Onerous reporting also increases the regulatory burden, an issue for both federal and state governments that can outweigh any benefits gained. 

When states are not on board, they tend to give `tick-a-box' responses, going through the motions of complying without actually enforcing real change. An international review highlighted that the lack of state government buy-in to Australia's COAG reform processes was a significant barrier to making the collaborative effort work.\footcite{HowesEngele2013WhytheCOAGReformAgendaHasFloundered, HowesEngele2013FederalReformStrategies}

A key lesson is that policy coherence can be compromised where policy levers are dispersed between Commonwealth and state governments. The involvement of an additional tier of government can create confusion for teachers, schools and systems, especially if policies and funding arrangements chop and change.

\section{Alternative forms of Commonwealth control do not have clear benefits}\label{sec:other-approaches}

Tying federal government funding to national outcomes-based indicators, with punishment for states that fail to meet targets, can also have damaging consequences. For example, the US `No Child Left Behind' policies led to some perverse outcomes for few gains (see \Vref{box:US-approaches-to-federal-accountability}).

Nor should the federal government establish new `quality assurance frameworks', given the fraught status of Commonwealth-state relations. The US federal government has adopted a quality assurance approach under The Every Student Succeeds Act (ESSA) (2015), demanding states show the extent to which their policies and programs are evidence-based. It is too early to assess its success (see \Vref{box:US-approaches-to-federal-accountability}). But in Australia, such an approach would increase red tape, and the states and territories would be likely to regard it as a threat to their turf.

\begin{bigbox}{US approaches to federal accountability}{box:US-approaches-to-federal-accountability}

\textbf{No Child Left Behind, 2001 onwards}

The No Child Left Behind (NCLB) policy was introduced with bi-partisan support in 2001, but had such calamitous consequences that there was also bipartisan support for its abolition in 2014.

Under NCLB, the federal government demanded that all states each year test 95 per cent of their students between years 3 and 8 in reading and maths. Schools were punished if they were judged not to have met their (state set) targets on `Adequate Yearly Progress' (AYP). Punitive actions included restructuring of the school, or allowing and encouraging parents to send their children to other schools.

This top-down, high-stakes testing approach created perverse incentives.\footnote{Discussed in \textcite[][39]{Goss2015TargetedTeachingHow}.}
For example, state governments began lowering their standards when setting their AYP targets. Teachers began narrowing the curriculum to prepare their students for the tests in reading and maths.\footcite{Jacob2010TheImpactofNoChildLeftBehind}
Some schools discouraged students from participating in the tests. Some schools and teachers resorted to desperate measures to alter test records.
By the end, there was unanimous support for reinventing federal policies and abandoning high-stakes testing.

\textbf{The Every Student Succeeds Act (ESSA) (2015)}

ESSA reduces federal control by moving away from demanding high-stakes accountability to demanding states show that their policies and programs are evidence-based.

Under ESSA, all states are still required to monitor and test 95 per cent of their students, but the federal government now allows states to develop their own system of accountability for underperforming schools.

States must show that they have evidence-based plans and interventions to address underperformance. There is an agreed national framework for what constitutes `evidence'.

ESSA reduces the punitive consequences for schools not making adequate progress, and will avoid most of the perverse incentives created under NCLB\@. But it is too early to say whether these policy changes will result in improved education outcomes in the US\@. 

\end{bigbox}

