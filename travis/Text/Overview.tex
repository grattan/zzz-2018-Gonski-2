\begin{overview}
Australia needs a new national conversation on school education. We should seize the opportunity provided by the Commonwealth's Review to Achieve Educational Excellence in Australian Schools (known as the `Gonski 2.0 Review'). 

The Turnbull Government commissioned the Gonski 2.0 Review in an effort to ensure the extra Commonwealth money going into schools over the next decade is spent wisely by the states and territories. But this report warns against over-reach: too much Commonwealth intervention into school education could be counterproductive and costly.

Under the Gonski 2.0 funding deal struck last year, schools will get an extra \$23 billion in Commonwealth funds over the next ten years. 
But the Commonwealth's need for reassurance about how the money is spent must be kept in perspective: the extra federal funding is only 3 per cent of all government spending on schools over the period. 

Much more important is that all government money for school education is spent effectively, regardless of where it comes from. This report first identifies the big system reforms needed to improve students outcomes. Most of these reforms are the responsibilities of state and territory governments. Then the report considers what few things the Commonwealth should do to help. 

The biggest advances will be made only if Australia adopts a more `adaptive' school education system. Neither a top-down nor a bottom-up model of governance is desirable. Instead, schools need more support to ensure teachers know what works in the classroom, and how they can adapt their teaching methods to better meet the needs of their students. This requires a much greater focus on student progress, and on how teachers use data to evaluate and target their teaching. Teachers need more opportunities to develop and get feedback from their colleagues, along with more guidance on tried and tested classroom materials to reduce `reinvention of the wheel'.

Driving any of these big reforms from Canberra would be difficult. Imposing prescriptive funding conditions on states and territories can destroy policy coherence and simply increase red tape. The Commonwealth has few ways to independently verify if change is actually happening in the classroom, and adding an extra layer of government policies that chop and change only disrupts schools and teachers. The Turnbull Government's 2016 Quality Schools Quality Outcomes policy -- which includes a long list of over 15 potential new national requirements -- is a big step in the wrong direction. 

Instead we recommend the Commonwealth focus strategically on a few national reforms. Given the difficulties in Commonwealth-state relations, it is far better to focus on a few actions with a high chance of success and strong buy-in from state governments. The Commonwealth should abandon any policy `reform' that does not meet all three criteria: evidence shows it is a good idea; government can make it happen; and Commonwealth intervention will help. 

We nominate four areas likely to succeed as national reforms: invest in measuring new, 21st century skills; develop betters ways to measure student progress; invest in high-quality digital assessment tools for teachers; and create a new national research organisation to share what works best.

The extra Commonwealth money for schools under Gonski 2.0 is welcome. But for Australian students to get the most benefit, the Commonwealth must resist the temptation to over-reach by intervening heavily in school education policy.

\end{overview}
