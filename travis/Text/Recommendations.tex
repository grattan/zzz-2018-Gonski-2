
\begin{recommendations}


%\chapter{Policy recommendations}\label{chap:Policy-recommendations}

In pursuit of national reform in school education, the Commonwealth Government should:

\begin{itemize}
    \item \textbf{First, deliver fully on existing Commonwealth responsibilities before embarking on new national initiatives.} The federal government's role in initial teacher education, delivering a rigorous national curriculum, improving national assessments, and embedding the professional standards all require constant attention, and some require urgent reform. 
    
    \item \textbf{Recognise that many of the big reforms are the responsibility of the system managers; that is, the state and territory governments.} The Commonwealth must collaborate with the states and territories on any new national efforts; state and territory `buy-in' is essential.
    
    \item \textbf{Select a small number of national reforms (only) and do them well.} Given the difficulties in driving improvement from Canberra, avoid spreading efforts too thinly. 
    
     \item \textbf{Prioritise the reforms most likely to succeed as national reforms.} We suggest three key criteria:
     
     
    (i) Is it a good idea?
    
    (ii) Can government make it happen?
    
    (iii) Will Commonwealth intervention help?
    
    \end{itemize}
   
\newpage
 
Specifically, the Commonwealth Government should consider the following four reforms that meet the above criteria:
    
    \begin{enumerate}
   
    \item \textbf{Invest nationally to improve how we measure non-cognitive and critical thinking skills.} There is a big need for research in this area, and it should be done in collaboration with states and territories.
    
    \item \textbf{Develop new national measures of learning progress} for diagnostic use in the classroom, and for national bench-marking in collaboration with state and territory curriculum and assessment authorities.
    
    \item \textbf{Invest in high-quality digital tools to help teachers regularly assess classroom learning} alongside a new `star rating' system to help schools when searching for the most appropriate assessment tools.
    
    \item \textbf{Establish a new national independent evidence body} to help identify national priorities, set rigorous standards of evidence, fund high-quality research (especially randomised controlled trials) and disseminate and promote findings. This body could link-up various strands of research on education for people from birth through to age 18.
    
   \end{enumerate}
   

\end{recommendations}


    
   






